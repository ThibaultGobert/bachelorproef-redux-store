\chapter{Stand van zaken}
\label{ch:stand-van-zaken}

% Tip: Begin elk hoofdstuk met een paragraaf inleiding die beschrijft hoe
% dit hoofdstuk past binnen het geheel van de bachelorproef. Geef in het
% bijzonder aan wat de link is met het vorige en volgende hoofdstuk.

% Pas na deze inleidende paragraaf komt de eerste sectiehoofding.
\section{React}
Volgens \autocite{React01} is React een JavaScript library om user interfaces te bouwen. React is component-based. Een component implementeert een \textit{render} methode die data als input neemt en een view retourneert. Om dit te doen wordt JSX gebruikt. Er kunnen simpele views gemaakt worden voor elke state van de applicatie en React zal deze updaten en de juiste componenten renderen wanneer de data verandert.

\subsection{JSX}
JSX is een XML/HTML-achtige syntax gebruikt door React die ECMAScript uitbreidt zodat XML/HTML-achtige text kan bestaan met JavaScript/React code. Deze syntax is bedoeld om gebruikt te worden door preprocessoren (zoals Babel) om HTML text gevonden in JavaScript files om te zetten in standaard JavaScript objecten. Dit wil dus zeggen dat je door JSX te gebruiken HTML structuren en JavaScript code kan schrijven in dezelfde file, Babel zal dan alle uitdrukkingen vertalen naar JavaScript code. Waar vroeger dus JavaScript code in HTML werd geplaatst, laat JSX het toe om HTML in JavaScript te plaatsen.
(TODO REF JSX)

\subsection{Components}
Een enkele view van een user interface is opgedeeld in een aantal stukken, in een aantal components. De start component bevat een tree, die tree kan opgedeeld worden in een aantal sub-componenten. Deze kunnen dan weer opgedeeld worden in nog meer sub-componenten. Dit kan dus resulteren in een complexe tree met verschillende React componenten.  
(TODO REF JSX ZELFDE)

\paragraph{Simple component}
React components implementeren de \textit{render} methode, ze retourneren wat er moet afgebeeld worden. Daarvoor wordt het eerder aangekaarte JSX gebruikt. Input data die doorgegeven is aan de component kan opgeroepen worden door de \textit{props} aan te spreken van deze component. Sidenote: JSX is optioneel, het gewenste resultaat kan ook bereikt worden door JavaScript code alleen. JSX maakt het wel overzichtelijker om de props aan te spreken.  
\autocite{React01}

\paragraph{Stateful component}
In toevoeging met data als input nemen (via \textit{props}), kan een component zijn interne state data aanspreken. Wanneer de state data van een component verandert, wordt de render methode opnieuw aangeropen zodat de juiste data getoond wordt.
\autocite{React01}

Dit hoofdstuk bevat je literatuurstudie. De inhoud gaat verder op de inleiding, maar zal het onderwerp van de bachelorproef *diepgaand* uitspitten. De bedoeling is dat de lezer na lezing van dit hoofdstuk helemaal op de hoogte is van de huidige stand van zaken (state-of-the-art) in het onderzoeksdomein. Iemand die niet vertrouwd is met het onderwerp, weet er nu voldoende om de rest van het verhaal te kunnen volgen, zonder dat die er nog andere informatie moet over opzoeken \autocite{Pollefliet2011}.

Je verwijst bij elke bewering die je doet, vakterm die je introduceert, enz. naar je bronnen. In \LaTeX{} kan dat met het commando \texttt{$\backslash${textcite\{\}}} of \texttt{$\backslash${autocite\{\}}}. Als argument van het commando geef je de ``sleutel'' van een ``record'' in een bibliografische databank in het Bib\TeX{}-formaat (een tekstbestand). Als je expliciet naar de auteur verwijst in de zin, gebruik je \texttt{$\backslash${}textcite\{\}}.
Soms wil je de auteur niet expliciet vernoemen, dan gebruik je \texttt{$\backslash${}autocite\{\}}. In de volgende paragraaf een voorbeeld van elk.

\textcite{Knuth1998} schreef een van de standaardwerken over sorteer- en zoekalgoritmen. Experten zijn het erover eens dat cloud computing een interessante opportuniteit vormen, zowel voor gebruikers als voor dienstverleners op vlak van informatietechnologie~\autocite{Creeger2009}.

\lipsum[7-20]
