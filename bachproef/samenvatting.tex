%%=============================================================================
%% Samenvatting
%%=============================================================================

% TODO: De "abstract" of samenvatting is een kernachtige (~ 1 blz. voor een
% thesis) synthese van het document.
%
% Deze aspecten moeten zeker aan bod komen:
% - Context: waarom is dit werk belangrijk?
% - Nood: waarom moest dit onderzocht worden?
% - Taak: wat heb je precies gedaan?
% - Object: wat staat in dit document geschreven?
% - Resultaat: wat was het resultaat?
% - Conclusie: wat is/zijn de belangrijkste conclusie(s)?
% - Perspectief: blijven er nog vragen open die in de toekomst nog kunnen
%    onderzocht worden? Wat is een mogelijk vervolg voor jouw onderzoek?
%
% LET OP! Een samenvatting is GEEN voorwoord!

%%---------- Nederlandse samenvatting -----------------------------------------
%
% TODO: Als je je bachelorproef in het Engels schrijft, moet je eerst een
% Nederlandse samenvatting invoegen. Haal daarvoor onderstaande code uit
% commentaar.
% Wie zijn bachelorproef in het Nederlands schrijft, kan dit negeren, de inhoud
% wordt niet in het document ingevoegd.

\IfLanguageName{english}{%
\selectlanguage{dutch}
\chapter*{Samenvatting}
\lipsum[1-4]
\selectlanguage{english}
}{}

%%---------- Samenvatting -----------------------------------------------------
% De samenvatting in de hoofdtaal van het document

\chapter*{\IfLanguageName{dutch}{Samenvatting}{Abstract}}

In dit onderzoek zal gezocht worden naar een oplossing om 2 React-Redux projecten en meer bepaald 2 stores met elkaar te integreren.  Hierbij wordt dus uitgegaan van 2 projecten. Het eerste project zal een basis inlog functionaliteit hebben. Voor het tweede project zal een open source project gebruikt worden en gepublished worden op npm onder een eigen registry. De bedoeling is dat het eerste project het tweede project zal gebruiken als npm-package. 

De bekende oplossing voor dit probleem is het kopiëren van de reducers van het tweede project en deze plaatsen in het eerste project. [TODO-REFERENTIE HIER]
Deze oplossing heeft echter een groot nadeel: als de reducers van het open source project gewijzigd worden of als er reducers worden toegevoegd/verwijderd, dan zullen er ook aanpassingen moeten gebeuren in het eerste project. 

De bedoeling van dit onderzoek is om een manier te vinden waarbij geen extra aanpassingen moeten gebeuren aan het eerste project, maar enkel een geupdatete versie van het tweede project moet gepublished worden op npm.
Deze manier zal verkregen worden door een extensie te schrijven op Redux die de mogelijheid biedt om de changes zonder aanpassingen in het eerste project te implementeren. 
