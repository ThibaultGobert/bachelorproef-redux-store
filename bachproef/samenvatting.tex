%%=============================================================================
%% Samenvatting
%%=============================================================================

% TODO: De "abstract" of samenvatting is een kernachtige (~ 1 blz. voor een
% thesis) synthese van het document.
%
% Deze aspecten moeten zeker aan bod komen:
% - Context: waarom is dit werk belangrijk?
% - Nood: waarom moest dit onderzocht worden?
% - Taak: wat heb je precies gedaan?
% - Object: wat staat in dit document geschreven?
% - Resultaat: wat was het resultaat?
% - Conclusie: wat is/zijn de belangrijkste conclusie(s)?
% - Perspectief: blijven er nog vragen open die in de toekomst nog kunnen
%    onderzocht worden? Wat is een mogelijk vervolg voor jouw onderzoek?
%
% LET OP! Een samenvatting is GEEN voorwoord!

%%---------- Nederlandse samenvatting -----------------------------------------
%
% TODO: Als je je bachelorproef in het Engels schrijft, moet je eerst een
% Nederlandse samenvatting invoegen. Haal daarvoor onderstaande code uit
% commentaar.
% Wie zijn bachelorproef in het Nederlands schrijft, kan dit negeren, de inhoud
% wordt niet in het document ingevoegd.

\IfLanguageName{english}{%
\selectlanguage{dutch}
\chapter*{Samenvatting}
\lipsum[1-4]
\selectlanguage{english}
}{}

%%---------- Samenvatting -----------------------------------------------------
% De samenvatting in de hoofdtaal van het document

\chapter*{\IfLanguageName{dutch}{Samenvatting}{Abstract}}

In dit onderzoek worden twee React-Redux projecten opgesteld. React maakt gebruik van JavaScript voor het bouwen van user interfaces, waar Redux een concept is voor data storage en communicatie binnen de applicatie. Een store is een concept van Redux die de hele state tree van de applicatie vasthoudt. Het probleem situeert zich in het samenvoegen van twee React-Redux projecten. Deze hebben beide een store die de overeenkomstige state tree bevat. Er kan dus niet rechtstreeks een component geïmporteerd worden van het ene project naar het andere, omdat het daar geen bekende state zal hebben en er zo geen acties op kunnen uitgevoerd worden. Om dit probleem te reproduceren worden dus twee React-Redux projecten opgesteld. Het eerste project is het hoofdproject  met een basis inlog functionaliteit. Het tweede project is een fork van het bestaande open source project van Scratch. Om beide projecten met elkaar te laten communiceren zal het tweede project gepubliceerd worden op npm onder een eigen registry. Dit zorgt ervoor dat het project geïnstalleerd kan worden als dependency in het eerste project. Op deze manier kan een component snel geïmporteerd worden. 

De bedoeling van dit onderzoek is om een manier te vinden waarbij geen extra aanpassingen moeten gebeuren aan het eerste project, maar enkel een geupdatete versie van het tweede project moet gepublished worden op npm. Deze manier zal verkregen worden door een extensie te schrijven op Redux (de functie die reducers combineert) die de mogelijkheid biedt om de functionaliteit van het tweede project te gebruiken in het eerste project. Tot dusver is er nog geen bekende manier om de combinatie van een root reducer ongedaan te maken en opnieuw te combineren met de reducers van alle projecten samen. Dit wil zeggen dat de enige bekende manier een export van de bestaande reducers is. Er is dus zeker nog ruimte voor verder onderzoek over hoe reeds gecombineerde reducers opnieuw gecombineerd kunnen worden met elkaar.
