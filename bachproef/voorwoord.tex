%%=============================================================================
%% Voorwoord
%%=============================================================================

\chapter*{Woord vooraf}
\label{ch:voorwoord}

%% TODO:
%% Het voorwoord is het enige deel van de bachelorproef waar je vanuit je
%% eigen standpunt (``ik-vorm'') mag schrijven. Je kan hier bv. motiveren
%% waarom jij het onderwerp wil bespreken.
%% Vergeet ook niet te bedanken wie je geholpen/gesteund/... heeft

Op mijn stage werd ik voor het eerst gevraagd om een React-Redux integratie uit te voeren. Ik werd hier snel geconfronteerd met een aantal problemen die niet gedocumenteerd waren of nergens beschreven stonden. Ik moest daar namelijk de integratie voorzien van het open source React-Redux project van Scratch. De reden waarom ik dit project ook heb gekozen om mijn bachelorproef uit te voeren, is omdat ik dit project het beste ken en daardoor zo weinig mogelijk tijd zou verliezen door mij te verdiepen in een ander project. Nu Scratch overgeschakeld is op JavaScript voor hun graphical user interface, wordt het zeer interessant om een makkelijke manier te voorzien om dit project te integreren. Zeker voor een bedrijf zoals CodeFever, waar ik stage heb gedaan. Zij voorzien namelijk programmeerlessen voor kinderen, dus het zou zeer handig zijn moesten zij een ingebouwde versie van Scratch hebben. Nu zijn ze afhankelijk van Scratch en met een geïntegreerde versie van Scratch zouden ze extra functionaliteit kunnen toevoegen. Nu staan hun oefeningen publiekelijk zichtbaar op de website van Scratch. Indien ze bij de geïntegreerde versie van Scratch zelf een opslag functie kunnen voorzien voor hun oefeningen waarbij deze weggeschreven worden in hun database kunnen ze ervoor zorgen dat hun oefeningen uniek gekoppeld zijn aan CodeFever.

Het werd voor mij al snel duidelijk dat er quasi niets van documentatie online te vinden was over het combineren of integreren van twee React-Redux projecten. Dit vond ik zeer raar, omdat het net zo belangrijk is voor CodeFever om dit te kunnen doen, dacht ik dat er zeker nog bedrijven geconfronteerd werden met dit probleem. Het zorgde er ook voor dat ik een oplossing voor dit probleem wou vinden. 

Verder wil ik Harm De Weirdt bedanken voor de begeleiding alsook Simon De Gheselle om mijn co-promotor te zijn. Finaal bedank ik de Hogeschool Gent om het mogelijk te maken deze scriptie te schrijven.