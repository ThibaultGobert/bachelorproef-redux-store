%%=============================================================================
%% Voorwoord
%%=============================================================================

\chapter*{Woord vooraf}
\label{ch:voorwoord}

%% TODO:
%% Het voorwoord is het enige deel van de bachelorproef waar je vanuit je
%% eigen standpunt (``ik-vorm'') mag schrijven. Je kan hier bv. motiveren
%% waarom jij het onderwerp wil bespreken.
%% Vergeet ook niet te bedanken wie je geholpen/gesteund/... heeft

Op mijn stage werd ik het eerst geconfronteerd met dit probleem. Ik moest daar namelijk de integratie voorzien van een react-redux project (Scratch). De reden waarom ik dit project ook heb gekozen is omdat ik dit project het beste ken en daardoor zo weinig mogelijk tijd zou verliezen door mij te verdiepen in een ander project. Nu scratch overgeschakeld is op JavaScript voor hun graphical user interface, wordt het zeer interessant om een makkelijke manier te voorzien om dit project te integreren. Zeker voor een bedrijf zoals CodeFever, waar ik stage heb gedaan. Zij voorzien namelijk programmeerlessen voor kinderen, dus het zou zeer handig zijn moesten zij een ingebouwde versie van Scratch hebben. 

Wat mij dan verontruste was dat er quasi niets van documentatie hierover te vinden was online. Dit vond ik zeer raar, omdat het net zo belangrijk is om dit te kunnen doen. Dit zorgde er ook voor dat ik een oplossing voor dit probleem wou vinden. 

Verder wil ik nog Harm De Weirdt bedanken voor de begeleiding alsook Simon De Gheselle om mijn co-promotor te zijn. Finaal bedank ik ook de Hogeschool Gent om het mogelijk te maken deze scriptie te schrijven.